\section{Design}

\subsection{Decision - Red-Black Tree}

After examining the time complexity for various dictionary implementations, I decided a red-black tree was the most effective implementation for a general-purpose dictionary.

\begin{table}[hp]
\centering
\begin{tabular}{| l | l | l | l |}
\hline
\textbf{Implementation} & \textbf{Search} & \textbf{Insert} & \textbf{Delete} \\ \hline
AVL tree & O(log n) & O(log n) & O(log n) \\ \hline
\color{red} Red-black tree & \color{red} O(log n) & \color{red} O(log n) & \color{red} O(log n) \\ \hline
B-Tree & O(log n) & O(log n) & O(log n) \\ \hline
Splay tree & (a) O(log n) & (a) O(log n) & (a) O(log n) \\ \hline
Skip list & O(n) & O(n) & O(n) \\ \hline
Trie & O(m) & O(m) & O(m) \\ \hline
Stratified tree & O(log log N) & O(log log N) & O(log log N) \\ \hline
\end{tabular}
\\{\tiny{(n = number of elements in data structure, m = length of string, N = size of universe, (a) = amortised)}}\\
\caption{Considered data structures along with their worst-case time complexities.}
\end{table}

When deciding on the optimal data structure, I first ruled out splay tree and skip list, since the skip list worst case time is above the required O(log n) and there are other data structures faster than the splay tree where amortisation is not required. I then ruled out Trie, although it is extremely fast (independent of size of dictionary), since I aimed to implement a general-purpose dictionary and Trie only holds strings \cite{trie}.

Initially I implemented basic functionality of a stratified tree \cite{stratified}, also known as a van Emde Boas tree. The implementation had extremely fast lookup (\code{contains}, \code{max}, \code{min} O(1), and \code{successor}, \code{predecessor}, \code{add}, \code{delete} O(log log N)). However, my primary aim for this project was a \textit{general-purpose} and \textit{unbounded} dictionary. van Emde Boas trees are extremely fast, however the domain is bounded and each element requires an associated integer key. In the end I decided that van Emde Boas trees are too restrictive, and also more suited for a dictionary interface defined using keys and values (like \code{java.util.Map.put(K key, V value)}) and thus a more general purpose and unbounded but less efficient O(log n) solution would be more appropriate for this situation.

A red-black tree and a B-Tree seem to be very similar in efficiency, especially since a red-black is simply a type of B-Tree, as implied by the red-black tree's original name: a symmetric binary B-Tree \cite{bayer}. It was challenging to find a significant advantage of red-black trees over B-Trees, however I chose to rule out B-Trees since they may waste space if a node is not full, whereas red-black trees do not have this disadvantage.

Finally, I chose a red-black tree over an AVL tree after considering the usage of the dictionary described in the project: keeping track of users logged in to a web-service. I assume that in most cases, more users log on and off than search for each other, so \textit{insertions and deletions} are more common than \textit{lookups}. Since red-black trees have generally faster inserts and deletions \cite{clrs} - whereas AVL trees are faster at searches with slower modifications - a red-black tree would be the most appropriate choice for this scenario.

\subsection{Implementation}

The majority of my implementation is adapted from the pseudocode in the book `Introduction to Algorithms: Third Edition' by T. H. Cormen, C. E. Leiserson, R. L. Rivest, and C. Stein. However, due to the specified requirements of the dictionary, my adaptation of the red-black tree has ended up with various modifications that differ significantly from the CLRS description. In this section, I detail the implementation of the \code{RedBlackTree} class.

\subsubsection{Searching}
I considered various options for the \code{contains()} method. My final implementation runs in logarithmic time by using the typical binary search tree search (traversing down the tree). However, I also considered the option of having a separate internal data structure such as a hash table. A hash table would provide \textit{much} faster lookups on average, but unfortunately still has a O(n) worst case. The project specifications require O(log n) worst case, and storing the data in a hash table as well would also essentially double the space requirement. In the end, I decided a basic tree search would suffice for this method.
 
\subsubsection{Successors/predecessors}
This part of the dictionary took several refactorings and was the most challenging part of the project to implement. At first I developed the methods to first locate the node with the given value, and then find the successor of that node using well-known binary search tree methods. Therefore initially, \code{hasPredecessor(Object)} and \code{hasSuccessor(Object)} would take logarithmic time, as they would first check whether the element was contained in the dictionary (O(log n)) and then check whether it was within the bounds of the dictionary (not $\leq$ min or $\geq$ max respectively). However after implementation I realised that it was not required that the argument be contained in the dictionary. Therefore \code{hasPredecessor(Object)} and \code{hasSuccessor(Object)} could be reduced to constant time - since the argument only has to be less than the maximum element to have a successor, and vice versa for predecessor. My final design for \code{successor(Object)} uses a method \code{above(Node)}, which finds the least node strictly greater than the argument (and \code{below(Node)} is symmetric for \code{predecessor(Object)}).

\subsubsection{Minimum/maximum}
Originally I implemented \code{max()} and \code{min()} to simply call the helper methods \code{minimum(Node)} and \code{maximum(Node)} with the root as the argument. This takes logarithmic time. However, I later modified the class to maintain two extra fields \code{min} and \code{max}. Upon insertion, the current values of \code{min} and \code{max} are compared with the newly inserted element - if the new node is greater than the max or less than the min, the references are updated. This adds a worst-case cost of 2  extra comparisons to the \code{add(Object)} method and reduces \code{min()} and \code{max()} to O(1).

Upon deletion, if the item being deleted is the minimum or the maximum, then the min/max is found by searching the tree again from the root. This is a minor cost when compared to the performance gains from constant time access to the extremum of the dictionary (and doesn't make the time complexity of \code{delete(Object)} any worse).

\subsubsection{Insertion/deletion} 

\begin{figure}[b]
\centering
\raisebox{-.5\height}{\includegraphics[width=0.25\textwidth]{resources/before}}
$\implies$
\raisebox{-.5\height}{\includegraphics[width=0.25\textwidth]{resources/after}}
\caption{Inserting characters \textit{a} to \textit{m} into the red-black tree, and then deleting the root \textit{d}. Note that in both cases, the height $h = 5$ and for a dictionary with 12-13 nodes the height bound is $h \leq 2\log_2(12+1) \approx 7$ which is satisfied by both structures.}
\end{figure}

The \code{add(Object)} and \code{delete(Object)} methods are where my implementation has strong similarities to the red-black tree detailed in CLRS.

The \code{add(Object)} method creates a new node and calls the \code{insert(Node)} method, which simply finds the appropriate location for the new node, inserts it there, colours it red, and then restores the red-black tree properties using \code{fixInsert(Node)}.

The \code{delete(Object)} method first locates the node to delete using a basic tree search, and replaces it with one of its children if necessary. Then it calls \code{fixDelete(Node)} to restore any red-black tree properties that may have been violated.

Since detailing how red-black tree insertion and deletion work would take more space than I have room for in this report, I will instead detail some implementation decisions used in regards to \code{add} and \code{delete} (the add- and delete-related code is commented in the \code{TreeIterator} class).

I chose to use a sentinel node \code{nil} for inserting and deleting, rather than null pointers. This is suggested by CLRS, and it has the advantage of not having to consider exceptional cases as frequently - instead of having to avoid dealing with null as a separate case, the \code{nil} node can be treated as any other node and have its \code{parent}, \code{left}, and \code{right} fields take on arbitrary values.

I felt one disadvantage to my insertion and deletion related code is that there is a lot of symmetrical/repeated code, that deals with the same case, but having every reference to a `left' child replaced with a `right' child, and vice versa. In development I was unable to find a clean way that would enable me to abstract this to deal with both left and right cases.

\subsubsection{Iteration}

In implementing the above methods, I tried to `modularise' my code as much as possible - that is, for example, instead of putting the deletion code inside the public \code{delete(Object)} method, I extracted it into a separate \code{delete(Node)} method. This gave me the advantage of being able to easily implement the iterator in a short amount of time, since I could rely on \code{successor(Node)} and \code{delete(Node)} to perform all operations needed by the \code{TreeIterator} class.

The iterator constructor takes in a start node, stores a reference to that start node in an instance variable, and then on each call to \code{next()}, `increments' the node by replacing it with its successor (using \code{successor(Node)}) and returns the previous value.

To implement \code{Iterator\#delete()}, I also stored a reference to the last node returned by \code{next()}. On deletion, I simply call the \code{delete(Node)} method in the enclosing \code{RedBlackTree} class, and then set \code{last} to the \code{nil} sentinel. Setting this field to \code{nil} also works as a check to ensure that the iterator state is valid - if \code{last} is \code{nil}, then we know that either \code{next()} hasn't been called yet or \code{last} has already been deleted. In this case, I throw an \code{IllegalStateException}.

Ensuring the iterator is fail-fast is also implemented in a fairly simple way - in the \code{RedBlackTree} class, a field \code{operations} keeps track of the total number of successful insertions and deletions on the dictionary. In the constructor of \code{TreeIterator}, this number is recorded. Then, each time a method is called on the iterator class, the two numbers are compared - if the number of operations on the tree is not the same as when the iterator was constructed, then a \code{ConcurrentModificationException} is immediately thrown.

\subsubsection{Logging}

Each action made on the dictionary will cause a new line to be appended to the log string. \code{getLogString()} will return this string and clear the log.

Each line of the log string reports the name of the method called, the value of the argument if applicable, and the number of comparisons used for that method.

Note that an assumption I have made in development is that the number of comparisons is the number of calls to \code{compareTo}, plus, in general, the number of significant colour comparisons (e.g. in \code{fixInsert(Node)} the number of times the parent node's colour is checked) and significant reference equality checks (e.g. moving down the tree in the \code{minimum(Node)} method.